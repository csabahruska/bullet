% Bullet-CB.tex
\begin{hcarentry}[updated]{Bullet}
\label{bullet}
\report{Csaba Hruska}%05/12
\status{experimental, active development}
\makeheader

Bullet is a professional open source multi-threaded 3D Collision
Detection and Rigid Body Dynamics Library written in C++. It is free
for commercial use under the zlib license. The Haskell bindings ship
their own (auto-generated) C compatibility layer, so the library can
be used without modifications. The Haskell binding provides a low
level API to access Bullet C++ class methods.  Some bullet classes
(Vector, Quaternion, Matrix, Transform) have their own Haskell
representation, others are binded as class pointers. The Haskell API
provides access to some advanced features, like constraints, vehicle
and more.

At the current state of the project most common services are
accessible from Haskell, i.e., you can load collision shapes and step
the simulation, define constraints, create raycast vehicle, etc.  More
advanced Bullet features (soft body simulation, Multithread and GPU
constaint solver, etc.) will be added later.

Currently we are developing a new high level FRP based API, which is
built top of Bullet.Raw module using the Elerea library.

%**<img width=500 src="./lc-stunts.png">
%*ignore
\begin{center}
\includegraphics[width=0.47\textwidth]{html/lc-stunts.png}
\end{center}
%*endignore

\FurtherReading
\url{http://www.haskell.org/haskellwiki/Bullet}
\end{hcarentry}
